\documentclass{beamer}
\title{git - the stupid content tracker}
\date{October 25, 2013}

\usefonttheme[onlylarge]{structurebold}
\setbeamerfont*{frametitle}{size=\normalsize,series=\bfseries}
\setbeamertemplate{navigation symbols}{}

\usepackage[english]{babel}
\usepackage{amsmath}
\usepackage{tabularx}
\usepackage{amsfonts}
\usepackage{float}
\usepackage{verbatim}
\usepackage{fancyhdr}
\usepackage{setspace}
\usepackage{listings}
\usepackage{color}

\usepackage[scaled]{beramono}

\definecolor{green}{HTML}{00bb00}

\lstset{
  basicstyle=\ttfamily\footnotesize,
  keywordstyle=\color[rgb]{0,0,1},
  commentstyle=\color[rgb]{0.133,0.545,0.133},
  stringstyle=\color[rgb]{1,0,1}
}

\newenvironment{myitemize}{
  \begin{itemize}
  \setlength{\itemsep}{5pt}
}{\end{itemize}}

\newcommand{\code}[1]{
  \lstinline[language=bash,basicstyle=\ttfamily]{#1}
}
  
\begin{document}

\begin{frame}
\titlepage
\end{frame}

\begin{frame}[t,fragile]{What is git?}
\begin{myitemize}
  \item<1->
    A revision control system that stores all history of your code in an
    immutable database.
  \item<2->
    Each repository is distributed. (No central servers!)
  \item<3->
    All interaction with a repository is done using the \code{git} command.
  \item<4->
    The man pages for \code{git} can be cryptic; Google or Stack Overflow is 
    usually the better option.
\end{myitemize}
\end{frame}

\begin{frame}[t,fragile]{Sharing is caring}
\begin{myitemize}
  \item<1->
    Two different types of repositories: \textbf{bare} and \textbf{not} bare.
  \item<2->
    Not bare repositories contain a working set of files for you to manage.
    Typically, only a \emph{single} person has access to a non bare repository.
  \item<3->
    Bare repositories contain a \code{git} database, but lack a working set
    of files.
    Bare repositories are typically shared with others.
  \item<4->
    We already have a lot of bare repositories in
    \code{/r/bcb/repositories}.
  \item<5->
    We share by \emph{pushing} and we take by \emph{pulling}.
\end{myitemize}
\end{frame}

\begin{frame}[t,fragile]{What is GitHub?}
\begin{myitemize}
  \item<1->
    Core feature: a place to store \textbf{a} bare repository that can be 
    shared with others.
  \item<2->
    Lots of other stuff: issue tracker, wiki, pretty interface, etc.
  \item<3->
    Useful for sharing your code with the world, but not necessary for
    collaborating with others in our department.
\end{myitemize}
\end{frame}

\begin{frame}[t,fragile]{What goes in a git repository?}
\begin{myitemize}
  \item<1->
    \textbf{Yes}: Source code, documentation, license, small data files.
  \item<2->
    \textbf{No}: Executables, large data files, archives.
  \item<3->
    General rules. They can be broken, but be careful: once a file is
    committed to your repository, it can be difficult to remove from your
    repository's database completely.
\end{myitemize}
\end{frame}

\begin{frame}[t,fragile]{Getting started}
\begin{myitemize}
  \item<1->
    Get a GitHub account.
  \item<2->
    Follow along at https://github.com/TuftsBCB/git-tutorial
\end{myitemize}
\end{frame}

\end{document}
